\chapter{Wprowadzenie}
\label{cha:wprowadzenie}

Projekt \textbf{AwareFit} powstał z przekonania, że większość dostępnych na rynku aplikacji treningowych to jedynie cyfrowe wersje papierowego notesu. Choć pozwalają one zapisać wykonane serie, rzadko dają użytkownikowi realną wartość dodaną w postaci analizy. W dzisiejszych czasach osoby trenujące siłowo wiedzą, że progres nie bierze się z przypadku, ale z precyzyjnego zarządzania objętością (tonażem), intensywnością oraz regeneracją. 

Głównym celem AwareFit było stworzenie inteligentnego systemu, który nie tylko archiwizuje jednostki treningowe, ale przede wszystkim je „rozumie”. Na rynku wciąż brakuje narzędzi, które potrafiłyby wyciągać wnioski z historii treningów i zamieniać surowe liczby w konkretne wskazówki. System AwareFit automatycznie przelicza progresję, analizuje tonaż i identyfikuje aktualny etap planu użytkownika. To aplikacja stworzona dla tych, którzy chcą trenować w pełni świadomie, co zresztą podkreśla sama jej nazwa.
System automatyzuje kluczowe procesy np. oblicza całkowitą objętość sesji, porównuje ją z wynikami z poprzednich tygodni, a dzięki autorskim algorytmom zaimplementowanym po stronie bazy danych, samodzielnie identyfikuje rodzaj realizowanego systemu treningowego. 

W warstwie technologicznej projekt skupia się na maksymalnym wykorzystaniu możliwości relacyjnej bazy danych PostgreSQL. Logika aplikacji nie spoczywa wyłącznie na barkach interfejsu, lecz jest głęboko zakorzeniona w procedurach składowanych i funkcjach bazodanowych (pgSQL). Takie podejście gwarantuje spójność danych, szybkość obliczeń oraz skalowalność systemu. AwareFit to kompletne środowisko, które łączy świat bazy danych, logiki serwerowej PHP oraz nowoczesnego interfejsu użytkownika, tworząc spójne narzędzie do świadomego kształtowania własnej formy fizycznej.