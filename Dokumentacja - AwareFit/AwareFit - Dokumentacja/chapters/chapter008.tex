\chapter{Instrukcja uruchomienia aplikacji}

Niniejszy rozdział opisuje procedurę konfiguracji środowiska lokalnego oraz kroki niezbędne do poprawnego uruchomienia aplikacji AwareFit.

\section{Wymagania systemowe i programowe}
Do poprawnego działania aplikacji wymagane jest zainstalowanie i skonfigurowanie następujących komponentów:
\begin{itemize}
    \item \textbf{XAMPP}: Serwer lokalny z aktywnym modułem Apache.
    \item \textbf{PostgreSQL}: Silnik bazy danych wraz z narzędziem do zarządzania \textbf{pgAdmin 4}.
    \item \textbf{PHP PDO PostgreSQL}: Rozszerzenie \texttt{php\_pdo\_pgsql} musi być odkomentowane w pliku konfiguracyjnym \texttt{php.ini}.
\end{itemize}

\section{Konfiguracja bazy danych}
W celu przygotowania warstwy danych należy wykonać poniższe kroki:
\begin{enumerate}
    \item Uruchomić narzędzie pgAdmin 4 i utworzyć nową bazę danych o nazwie \texttt{awarefit\_db}.
    \item Zaimportować strukturę bazy danych oraz funkcje systemowe z dostarczonego pliku SQL.
    \item Parametry połączenia:
    \begin{itemize}
        \item \textbf{Użytkownik}: \texttt{postgres}
        \item \textbf{Hasło}: \texttt{psql} (Uwaga: W przypadku instalacji na innym serwerze należy zaktualizować hasło w pliku konfiguracyjnym aplikacji).
    \end{itemize}
\end{enumerate}

\section{Instalacja i uruchomienie}
Aplikacja jest zaimplementowana jako serwis webowy i powinna być hostowana wewnątrz środowiska XAMPP.
\begin{enumerate}
    \item Folder z kodem źródłowym (zawierający pliki PHP oraz interfejs graficzny GUI) należy umieścić w katalogu: \\
    \texttt{C:\textbackslash xampp\textbackslash htdocs\textbackslash Projekt\_Bazy\_Danych\textbackslash}
    \item W panelu kontrolnym XAMPP należy uruchomić usługę \textbf{Apache}.
    \item Aplikacja jest dostępna pod adresem lokalnym:
    \begin{quote}
        \url{http://localhost/Projekt_Bazy_Danych/index.php}
    \end{quote}
\end{enumerate}

\newpage
\section{Konfiguracja połączenia z bazą danych (PHP)}
Aplikacja wykorzystuje sterownik \textbf{PDO} (\textit{PHP Data Objects}) do komunikacji z bazą PostgreSQL. Kluczowe parametry połączenia zdefiniowane są w pliku konfiguracyjnym:
\begin{quote}
    \texttt{include/db.php}
\end{quote}

W przypadku zmiany parametrów serwera (np. przeniesienia bazy na inny host lub zmiany hasła systemowego), należy zaktualizować poniższy fragment kodu:

\begin{lstlisting}[language=PHP, caption={Zawartość pliku include/db.php}, frame=single]
<?php
$host = "localhost";
$port = "5432";
$dbname = "awarefit_db";
$user = "postgres";
$password = "psql";

try {
    $dsn = "pgsql:host=$host;port=$port;dbname=$dbname";
    $pdo = new PDO($dsn, $user, $password, [
        PDO::ATTR_ERRMODE => PDO::ERRMODE_EXCEPTION
    ]);
} catch (PDOException $e) {
    die("Błąd połączenia: " . $e->getMessage());
}
?>
\end{lstlisting}

\subsection{Kluczowe zmienne konfiguracyjne}
\begin{itemize}
    \item \texttt{\$host}: Adres serwera bazy danych (domyślnie \texttt{localhost}).
    \item \texttt{\$dbname}: Nazwa docelowej bazy danych (\texttt{awarefit\_db}).
    \item \texttt{\$password}: Hasło użytkownika PostgreSQL (w tej konfiguracji ustawione na \texttt{psql}).
    \item \texttt{PDO::ATTR\_ERRMODE}: Ustawienie \texttt{ERRMODE\_EXCEPTION} wymusza rzucanie wyjątków w przypadku błędnych zapytań SQL, co ułatwia debugowanie aplikacji.
\end{itemize}

\newpage
\section{Dostęp do konta testowego}
W celu poprawnej weryfikacji wszystkich funkcjonalności systemu bazodanowego (takich jak obliczanie objętości treningowej, generowanie wykresów progresji czy automatyczne wykrywanie rodzaju splitu), w bazie danych zostało przygotowane dedykowane konto testera. 

Konto to zawiera historyczne dane treningowe niezbędne do poprawnego renderowania statystyk. Aby wejść do aplikacji, należy użyć poniższych danych uwierzytelniających:

\begin{itemize}
    \item \textbf{Nazwa użytkownika}: \texttt{tester}
    \item \textbf{Hasło}: \texttt{test}
\end{itemize}

\section{Instrukcja generowania wykresu}
Po zalogowaniu na konto \texttt{tester}, użytkownik uzyskuje dostęp do głównego panelu sterowania (\textit{Dashboard}). Aby zweryfikować poprawność działania dynamicznych wykresów progresji objętościowej, należy przeprowadzić procedurę wyboru ćwiczenia zgodnie z historią treningową zapisaną w bazie danych.

Wykresy generowane są na podstawie unikalnych par: partia mięśniowa oraz konkretne ćwiczenie. Aby zobaczyć wykres, należy w panelu wybrać:

\begin{itemize}
    \item \textbf{Partia mięśniowa}: \texttt{Klatka piersiowa}
    \item \textbf{Ćwiczenie}: \texttt{Wyciskanie sztangi na ławce płaskiej}
\end{itemize}
