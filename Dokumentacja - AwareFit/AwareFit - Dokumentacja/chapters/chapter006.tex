\chapter{Dostęp zdalny do bazy danych}

W niniejszym rozdziale przedstawiono koncepcję oraz techniczną realizację komunikacji pomiędzy aplikacją AwareFit a systemem zarządzania bazą danych PostgreSQL.

\section{Koncepcja dostępu zdalnego}

\textbf{Model architektury systemowej:} \\
Aplikacja AwareFit została zaprojektowana jako progresywna platforma webowa działająca w modelu \textit{Client-Server}. Koncepcja dostępu zdalnego opiera się na separacji interfejsu użytkownika od fizycznej lokalizacji danych.

\begin{itemize}
    \item \textbf{Dostęp przez protokół HTTP/HTTPS:} Użytkownik uzyskuje dostęp do aplikacji poprzez przeglądarkę internetową. Eliminuje to konieczność instalacji bazy danych na urządzeniu końcowym – wszystkie operacje są procesowane zdalnie na serwerze.
    \item \textbf{Rola serwera pośredniczącego:} Serwer Apache (XAMPP) pełni rolę bramy (Gateway). Przyjmuje on żądania od użytkownika, a następnie nawiązuje zdalne połączenie z silnikiem PostgreSQL, aby pobrać lub zapisać dane.
    \item \textbf{Niezależność lokalizacji:} Dzięki zastosowaniu standardu TCP/IP oraz sterowników PDO, system jest przygotowany do pracy w rozproszonym środowisku sieciowym. Baza danych może znajdować się na dedykowanym hostingu, podczas gdy pliki strony serwowane są z innej lokalizacji.
\end{itemize}

\textbf{Zalety koncepcji webowej:} \\
Dzięki takiemu podejściu, dostęp zdalny zapewnia:
\begin{itemize}
    \item \textbf{Mobilność:} Użytkownik może edytować swój trening na telefonie w siłowni, a następnie analizować postępy na komputerze domowym – dane są zawsze zsynchronizowane w centralnej bazie.
    \item \textbf{Bezpieczeństwo:} Kluczowe dane (hasła, wyniki) nigdy nie są przechowywane bezpośrednio w przeglądarce, a jedynie przesyłane bezpiecznym kanałem do bazy danych.
\end{itemize}

\newpage
\section{Opis realizacji dostępu zdalnego}

Realizacja połączenia opiera się na rozszerzeniu \texttt{PDO} (PHP Data Objects). Środowisko uruchomieniowe zostało oparte na pakiecie \textbf{XAMPP} (serwer Apache) oraz systemie \textbf{PostgreSQL} zarządzanym przez narzędzie \textbf{pgAdmin 4}.

\textbf{Implementacja techniczna połączenia:} \\
Kluczowym elementem jest plik \texttt{includes/db.php}, który inicjuje sesję komunikacyjną. Poniżej przedstawiono kod źródłowy realizujący to zadanie:

\begin{verbatim}
<?php
$host = "localhost";
$port = "5432";
$dbname = "awarefit_db";
$user = "postgres";
$password = "psql";

try {
    $dsn = "pgsql:host=$host;port=$port;dbname=$dbname";
    $pdo = new PDO($dsn, $user, $password, [
        PDO::ATTR_ERRMODE => PDO::ERRMODE_EXCEPTION
    ]);
} catch (PDOException $e) {
    die("Błąd połączenia: " . $e->getMessage());
}
?>
\end{verbatim}

\textbf{Autoryzacja dostępu w warstwie aplikacji:} \\
Aby zapewnić, że zdalny dostęp do danych jest bezpieczny, każdy skrypt wymagający połączenia z bazą chroniony jest przez mechanizm sesji zawarty w pliku \texttt{auth.php}:

\begin{verbatim}
<?php
session_start();
if (!isset($_SESSION['user\_id'])) {
    header("Location: index.php");
    exit();
}
?>
\end{verbatim}

\textbf{Prezentacja i monitoring (pgAdmin 4):} \\
Do monitorowania stanu połączeń oraz weryfikacji poprawności zapisu danych wykorzystywane jest narzędzie \textbf{pgAdmin 4}. Pozwala ono na podgląd aktywnych procesów serwera oraz zarządzanie strukturą tabel (Rys. \ref{fig:pgadmin_structure}).

\begin{figure}[H]
    \centering
    \includegraphics[width=0.8\textwidth]{figures/GUI/pgAdmin4.jpg}
    \caption{Struktura bazy danych awarefit\_db widoczna w narzędziu pgAdmin 4}
    \label{fig:pgadmin_structure}
\end{figure}



\textbf{Weryfikacja działania:} \\
Poprawność realizacji dostępu zdalnego została potwierdzona poprzez:
\begin{enumerate}
    \item \textbf{Testy połączenia:} Weryfikacja reakcji aplikacji na błędne dane logowania (poprawne przechwycenie wyjątku \texttt{PDOException}).
    \item \textbf{Integracja z Apache:} Poprawne procesowanie żądań przez serwer Apache wewnątrz środowiska XAMPP i stabilne utrzymywanie połączenia z procesem \texttt{postgres.exe}.
    \item \textbf{Zdalne zarządzanie:} Możliwość edycji i przeglądu danych w czasie rzeczywistym bezpośrednio w interfejsie graficznym pgAdmin 4.
\end{enumerate}