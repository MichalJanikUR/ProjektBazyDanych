\chapter{Opis struktury bazy danych}
\label{cha:opisStrukturyBazyDanych}

\section{Zdefiniowane struktury tabel}

\subsection*{Tabela: users}
Tabela ta stanowi centralny punkt bazy danych, przechowując dane identyfikacyjne użytkowników oraz informacje niezbędne do procesów autoryzacji i personalizacji.

\begin{table}[h!]
\centering
\begin{tabular}{|l|l|p{7cm}|}
\hline
\textbf{Nazwa kolumny} & \textbf{Typ danych} & \textbf{Opis i ograniczenia} \\ \hline
id (PK) & serial & Unikalny identyfikator użytkownika. \\ \hline
username & varchar(50) & Nazwa użytkownika wykorzystywana do logowania. \\ \hline
password & varchar(255) & Zahaszowane hasło użytkownika. \\ \hline
email & varchar(100) & Adres e-mail przypisany do konta. \\ \hline
first\_name & varchar(50) & Imię użytkownika. \\ \hline
last\_name & varchar(50) & Nazwisko użytkownika. \\ \hline
gender & text & Płeć użytkownika. \\ \hline
\end{tabular}
\caption{Struktura tabeli users}
\end{table}

\subsection*{Tabela: body\_measurements}
Przechowuje historię pomiarów antropometrycznych użytkownika oraz zdefiniowane przez niego cele i poziomy aktywności.

\begin{table}[h!]
\centering
\begin{tabular}{|l|l|p{7cm}|}
\hline
\textbf{Nazwa kolumny} & \textbf{Typ danych} & \textbf{Opis i ograniczenia} \\ \hline
id (PK) & serial & Unikalny identyfikator pomiaru. \\ \hline
user\_id (FK) & integer & Powiązanie z tabelą users. \\ \hline
date & timestamp & Data i godzina wykonania pomiaru. \\ \hline
height & double precision & Wzrost użytkownika. \\ \hline
weight & double precision & Masa ciała. \\ \hline
chest & double precision & Obwód klatki piersiowej. \\ \hline
waist & double precision & Obwód pasa. \\ \hline
biceps & double precision & Obwód ramienia. \\ \hline
thighs & double precision & Obwód uda. \\ \hline
hips & double precision & Obwód bioder. \\ \hline
neck & numeric & Obwód szyi. \\ \hline
goal & varchar(50) & Cel treningowy (np. redukcja, masa). \\ \hline
activity\_level & numeric(4,3) & Współczynnik aktywności fizycznej. \\ \hline
\end{tabular}
\caption{Struktura tabeli body\_measurements}
\end{table}

\subsection*{Tabela: muscle\_groups}
Słownikowa tabela zawierająca nazwy partii mięśniowych, co pozwala na kategoryzację ćwiczeń.

\begin{table}[h!]
\centering
\begin{tabular}{|l|l|p{7cm}|}
\hline
\textbf{Nazwa kolumny} & \textbf{Typ danych} & \textbf{Opis i ograniczenia} \\ \hline
id (PK) & serial & Identyfikator grupy mięśniowej. \\ \hline
name & varchar(50) & Nazwa partii (np. klatka piersiowa, plecy). \\ \hline
\end{tabular}
\caption{Struktura tabeli muscle\_groups}
\end{table}

\subsection*{Tabela: exercises}
Katalog dostępnych w systemie ćwiczeń wraz z ich opisami i przypisaniem do konkretnych partii.

\begin{table}[h!]
\centering
\begin{tabular}{|l|l|p{7cm}|}
\hline
\textbf{Nazwa kolumny} & \textbf{Typ danych} & \textbf{Opis i ograniczenia} \\ \hline
id (PK) & serial & Identyfikator ćwiczenia. \\ \hline
name & varchar(100) & Nazwa ćwiczenia. \\ \hline
description & text & Instrukcja lub opis techniki ćwiczenia. \\ \hline
muscle\_group\_id (FK) & integer & Powiązanie z tabelą muscle\_groups. \\ \hline
\end{tabular}
\caption{Struktura tabeli exercises}
\end{table}

\newpage
\subsection*{Tabela: workouts}
Rejestruje każdą rozpoczętą sesję treningową użytkownika wraz z czasem jej trwania.

\begin{table}[h!]
\centering
\begin{tabular}{|l|l|p{7cm}|}
\hline
\textbf{Nazwa kolumny} & \textbf{Typ danych} & \textbf{Opis i ograniczenia} \\ \hline
id (PK) & serial & Identyfikator sesji treningowej. \\ \hline
user\_id (FK) & integer & Identyfikator użytkownika wykonującego trening. \\ \hline
date & timestamp & Data i godzina rozpoczęcia treningu. \\ \hline
duration & interval & Czas trwania sesji treningowej. \\ \hline
\end{tabular}
\caption{Struktura tabeli workouts}
\end{table}

\subsection*{Tabela: workout\_exercises}
Tabela pośrednicząca, która przypisuje konkretne ćwiczenia do danej jednostki treningowej.

\begin{table}[h!]
\centering
\begin{tabular}{|l|l|p{7cm}|}
\hline
\textbf{Nazwa kolumny} & \textbf{Typ danych} & \textbf{Opis i ograniczenia} \\ \hline
id (PK) & serial & Identyfikator wpisu ćwiczenia w treningu. \\ \hline
workout\_id (FK) & integer & Powiązanie z konkretną sesją (workouts). \\ \hline
exercise\_id (FK) & integer & Powiązanie z katalogiem ćwiczeń (exercises). \\ \hline
\end{tabular}
\caption{Struktura tabeli workout\_exercises}
\end{table}

\subsection*{Tabela: workout\_sets}
Najbardziej szczegółowa tabela systemu, przechowująca parametry każdej wykonanej serii.

\begin{table}[h!]
\centering
\begin{tabular}{|l|l|p{7cm}|}
\hline
\textbf{Nazwa kolumny} & \textbf{Typ danych} & \textbf{Opis i ograniczenia} \\ \hline
id (PK) & serial & Identyfikator serii. \\ \hline
workout\_exercise\_id (FK) & integer & Powiązanie z ćwiczeniem w ramach treningu. \\ \hline
weight & double precision & Obciążenie użyte w serii. \\ \hline
reps & integer & Liczba wykonanych powtórzeń. \\ \hline
set\_number & integer & Numer porządkowy serii w danym ćwiczeniu. \\ \hline
\end{tabular}
\caption{Struktura tabeli workout\_sets}
\end{table}

\newpage
\section{Powiązania pomiędzy tabelami i ich interpretacja}

Analiza diagramu ERD oraz struktur tabel pozwala na zdefiniowanie relacji, które gwarantują spójność danych oraz umożliwiają zaawansowaną analitykę treningową. Poniżej przedstawiono szczegółową interpretację kluczowych powiązań:

\begin{itemize}
    \item \textbf{Relacja users -- body\_measurements (1:N):} 
    To powiązanie pozwala na przypisanie wielu pomiarów ciała do jednego profilu użytkownika. Z punktu widzenia systemu jest to kluczowe dla monitorowania zmian w czasie (np. spadku wagi czy wzrostu obwodów). Dzięki tej relacji dashboard może generować wykresy trendów, porównując najnowszy wpis z danymi historycznymi.

    \item \textbf{Relacja users -- workouts (1:N):} 
    Fundament rejestracji aktywności. Każdy trening jest trwale przypisany do konkretnego użytkownika poprzez klucz obcy \textit{user\_id}. Umożliwia to izolację danych i obliczanie statystyk takich jak aktualna passa treningowa (streak) czy całkowita objętość podniesiona przez daną osobę w skali tygodnia.

    \item \textbf{Struktura hierarchiczna treningu (workouts -- workout\_exercises -- workout\_sets):} 
    Zastosowano tutaj potrójne powiązanie typu jeden-do-wielu, co odzwierciedla naturalną strukturę sesji na siłowni. Jeden rekord w \textit{workouts} (sesja) zawiera wiele wpisów w \textit{workout\_exercises} (wykonane ćwiczenia), a każde z tych ćwiczeń posiada wiele rekordów w \textit{workout\_sets} (serie). Taka dekompozycja danych pozwala na precyzyjne wyliczanie tonażu (objętości) dla każdego ćwiczenia z osobna.

    \item \textbf{Relacja exercises -- workout\_exercises (1:N):} 
    Łączy konkretne wykonanie ćwiczenia z jego definicją w katalogu. Dzięki temu system "wie", jakie ćwiczenie wykonuje użytkownik i może pobrać z tabeli \textit{exercises} jego opis lub przypisaną grupę mięśniową.

    \item \textbf{Relacja muscle\_groups -- exercises (1:N):} 
    Każde ćwiczenie jest sklasyfikowane pod jedną grupą mięśniową. To powiązanie jest niezbędne do działania algorytmu balansu strukturalnego. System sumuje serie z tabeli \textit{workout\_sets}, przechodzi przez powiązania do \textit{muscle\_groups} i na tej podstawie wylicza procentowy udział treningu klatki piersiowej, pleców czy nóg w skali tygodnia.
\end{itemize}

\newpage
\subsection{Opcjonalność relacji}
Dla zachowania spójności danych, w systemie zdefiniowano następujące zasady opcjonalności:

\begin{itemize}
    \item \textbf{Relacja users -- workouts (Obowiązkowa):} Każdy rekord w tabeli \textit{workouts} musi posiadać przypisane \textit{user\_id}. Nie jest możliwe istnienie treningu w systemie bez przypisanego autora (użytkownika).
    
    \item \textbf{Relacja workouts -- workout\_exercises (Obowiązkowa):} Każdy wpis o wykonanym ćwiczeniu musi odwoływać się do konkretnej sesji treningowej. Usunięcie treningu powoduje kaskadowe usunięcie przypisanych do niego ćwiczeń.
    
    \item \textbf{Relacja exercises -- muscle\_groups (Obowiązkowa):} W systemie każde ćwiczenie musi być skategoryzowane pod konkretną partią mięśniową, aby algorytmy analityczne mogły poprawnie przeliczać objętość tygodniową.
    
    \item \textbf{Relacja users -- body\_measurements (Opcjonalna):} Użytkownik może, ale nie musi posiadać wpisów w tabeli pomiarów. System dopuszcza istnienie konta bez historii pomiarów, choć ogranicza to wtedy dostęp do niektórych funkcji dashboardu (np. wykresów zmiany wagi).
    
    \item \textbf{Relacja workout\_exercises -- workout\_sets (Obowiązkowa):} Seria nie może istnieć bez powiązania z konkretnym ćwiczeniem w ramach danego treningu. 
\end{itemize}

\section{Diagram związków encji (ERD)}
Poniższy diagram przedstawia graficzną reprezentację struktury bazy danych systemu. Ilustruje on wszystkie opisane wcześniej tabele, ich atrybuty oraz relacje zachodzące pomiędzy poszczególnymi encjami, z uwzględnieniem kluczy głównych (PK) oraz obcych (FK).

\begin{figure}[h!]
    \centering
    \includegraphics[width=\textwidth]{figures/diagramERD.jpg}
    \caption{Diagram związków encji (ERD) projektowanej bazy danych.}
    \label{fig:diagram_erd}
\end{figure}

\newpage
\section{Kod implementujący strukturę bazy danych}
Poniższy przedstawiono kod, który posłużył do fizycznego utworzenia struktury tabel bazy danych zgodnie z zaprojektowanym diagramem ERD.
\begin{lstlisting}[language=SQL, caption={Kod implementujący strukturę tabel bazy danych}, breaklines=true, basicstyle=\small\ttfamily]
-- 1. Tabela uzytkownikow
CREATE TABLE public.users (
    id SERIAL PRIMARY KEY,
    username VARCHAR(50) NOT NULL UNIQUE,
    password VARCHAR(255) NOT NULL,
    email VARCHAR(100) NOT NULL UNIQUE,
    first_name VARCHAR(50),
    last_name VARCHAR(50),
    gender TEXT
);

-- 2. Tabela pomiarow ciala
CREATE TABLE public.body_measurements (
    id SERIAL PRIMARY KEY,
    user_id INTEGER NOT NULL,
    date TIMESTAMP WITHOUT TIME ZONE NOT NULL,
    height DOUBLE PRECISION,
    weight DOUBLE PRECISION,
    chest DOUBLE PRECISION,
    waist DOUBLE PRECISION,
    biceps DOUBLE PRECISION,
    thighs DOUBLE PRECISION,
    hips DOUBLE PRECISION,
    neck NUMERIC,
    goal VARCHAR(50),
    activity_level NUMERIC(4,3),
    FOREIGN KEY (user_id) REFERENCES public.users(id)
);

-- 3. Tabela grup miesniowych
CREATE TABLE public.muscle_groups (
    id SERIAL PRIMARY KEY,
    name VARCHAR(50) NOT NULL
);

-- 4. Tabela cwiczen
CREATE TABLE public.exercises (
    id SERIAL PRIMARY KEY,
    name VARCHAR(100) NOT NULL,
    description TEXT,
    muscle_group_id INTEGER NOT NULL,
    FOREIGN KEY (muscle_group_id) REFERENCES public.muscle_groups(id)
);

-- 5. Tabela treningow (naglowki)
CREATE TABLE public.workouts (
    id SERIAL PRIMARY KEY,
    user_id INTEGER NOT NULL,
    date TIMESTAMP WITHOUT TIME ZONE NOT NULL,
    duration INTERVAL,
    FOREIGN KEY (user_id) REFERENCES public.users(id)
);

-- 6. Tabela cwiczen w ramach treningu
CREATE TABLE public.workout_exercises (
    id SERIAL PRIMARY KEY,
    workout_id INTEGER NOT NULL,
    exercise_id INTEGER NOT NULL,
    FOREIGN KEY (workout_id) REFERENCES public.workouts(id),
    FOREIGN KEY (exercise_id) REFERENCES public.exercises(id)
);

-- 7. Tabela serii treningowych
CREATE TABLE public.workout_sets (
    id SERIAL PRIMARY KEY,
    workout_exercise_id INTEGER NOT NULL,
    weight DOUBLE PRECISION,
    reps INTEGER,
    set_number INTEGER,
    FOREIGN KEY (workout_exercise_id) REFERENCES public.workout_exercises(id)
);
\end{lstlisting}